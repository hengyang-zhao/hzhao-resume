\documentclass{resume} % Use the custom resume.cls style

\usepackage[left=0.8in,top=1.5in,right=0.8in,bottom=1.2in]{geometry} % Document margins

\name{Hengyang Zhao} % Your name
\address{Department of Electrical and Computer Engineering, University of California, Riverside} % Your address
\address{900 University Avenue, Riverside, CA 92507} % Your address
\address{+1~$\cdot$~(951)~$\cdot$~323~$\cdot$~9833 \\ hzhao@ece.ucr.edu} % Your phone number and email

\begin{document}

%----------------------------------------------------------------------------------------
%	EDUCATION SECTION
%----------------------------------------------------------------------------------------

\begin{rSection}{Education}

{\bf University of California, Riverside} \hfill {\em September 2014 -- present} \\
Ph.D. Candidate, in Electrical and Computer Engineering \\
Advisor: Dr. Sheldon X.-D. Tan

{\bf Shanghai Jiao Tong University} \hfill {\em September 2007 -- March 2014} \\
M.S., Instrument and Meter Engineering\\
B.S., Computer Science

\end{rSection}

%----------------------------------------------------------------------------------------
%	WORK EXPERIENCE SECTION
%----------------------------------------------------------------------------------------

\begin{rSection}{Experience}

    \begin{rSubsection}{Learning Based Electrical Vehicle Power Modeling}{February 2016 -- present}{Research Assistant}{UC Riverside, Collaboration with KAIST}

    \item Electrical vehicle power modeling based on feed-forward neural network.
    \item Neural network input preprocessing based on simple physical knowledge.

    \end{rSubsection}

    \begin{rSubsection}{Research on Smart Building Energy Reduction with Special Focus on Learning-Based Techniques}{March 2015 -- present}{Research Assistant}{UC Riverside}

    \item Recurrent neural network based approximate thermal modeling in smart building applications.

    \item People occupancy estimation based on analysis of sensor output.

    \item Sensor outlier/offset/fault detection using learning and probabilistic techniques.

    \end{rSubsection}

    \begin{rSubsection}{Research on GPU-Based Matrix LU Factorization (Direct Approach) for Circuit Simulation}{September 2014 -- Feburary 2015}{Research Assistant}{UC Riverside}

    \item Development of a fine grained parallel approach of GPU-based matrix LU factorization algorithm.
    \item Design Automation Conference 2015 Poster Session (Richard Newton Young Fellowship Program).

    \end{rSubsection}


    \begin{rSubsection}{Internship at Intel Inc.}{July 2013 –- August 2014}{Software Engineer}{Shanghai}

    \item Developed tool for automatically testing/profiling run-time data on a
        mobile operating system$*$. (hidden product name: ``*'')

    \item Developed an auxiliary tool to inspect the migration of the
        relationship between browser$*$ thread and the corresponding CPU* core's$*$ status within an interested duration.

    \end{rSubsection}

    \begin{rSubsection}{FPGA Based Capsule Endoscopy}{February 2012 -– March 2014}{FPGA/Verilog Developer}{SJTU}

    \item Participated in the design of wireless capsule endoscopy, including an
        FPGA-based swallow-able electronic capsule, a wireless data receiver and PC
        software.

    \item Implemented Verilog algorithm of color image baseline JPEG on the
        capsule-end Xilinx FPGA.

    \item Worked on the digital communication between the capsule endoscopy and
        the data receiver.

    \end{rSubsection}

    \begin{rSubsection}{Data Management System\\ at Sayes Medical Technology Co., LTD}{September 2012 -- January 2013}{Team Leader}{Shiyuan Inc., Shanghai}

    \item Developed a data management system for managing, browsing, processing
        and backing up the gastrointestinal data of PH, pressure and
        temperature records captured by electronic capsules.

    \item The management system was based on the server-client model, with one
        centralized Microsoft SQL Server database and multiple PC clients.

    \end{rSubsection}

    \begin{rSubsection}{Internship at Cisco Systems Inc.}{December 2011 -- June 2012}{Testing Engineer}{Shanghai}

    \item Participated in the automatic sanity test and duration test of Cisco
        phone models$∗$ . The actual testing work was to use Tcl script to
        setup test servers for automatically testing a large amount of phones.

    \item Maintained two Linux testing servers and resident guest virtual
        machines.

    \item Developed and maintained auxiliary scripts/tools to help debugging
        the testing scripts in Tcl/Tk.

    \end{rSubsection}

    \begin{rSubsection}{Implantable Physiological Parameters Detector}{December 2010 -- June 2011}{Hardware \& Software Designer}{SJTU}

    \item The animal physiological parameters detector system consists of a
        miniature implantable detector for measuring and transmitting ECG and
        blood pressure and body temperature information, and a hand-held
        wireless receiver.

    \item Designed and implemented the wireless receiver, supporting real-time
        ECG plotting, SD card storage and USB communication.

    \item Won the 3rd prize of outstanding graduation design in Dept. of Computer
        Science \& Engineering, SJTU.

    \end{rSubsection}

    \begin{rSubsection}{Undergraduate Innovation Project}{October 2009 -- September 2010}{Team Leader}{Shanghai}

    \item Designed an LED based, distributed intellectual lighting system. This
        system was a distributed network of independent lighting nodes with
        passive infrared sensors. Nodes negotiate and optimize the overall
        power consumption according to their different lighting demands
        acquired from the infrared sensors.

    \item Designed an 100W current-controlled, bulk-type switching power
        supply.

    \item This project was sponsored CNY 10000 by Shanghai government.

    \end{rSubsection}

    \begin{rSubsection}{IEEE Standard Micromouse Contest}{October 2009}{Team Leader}{Shanghai}

    \item Designed an micromouse, equipping one ARM Cortex-M3 micro controller,
        five infrared sensors and two stepper motors.

    \item The micromouse was placed in and was supposed to solve a IEEE
        standard 16 by 16 sized maze.  Our team won the 2nd prize in the
        contest of Yangtze River delta division.

    \end{rSubsection}

\end{rSection}

%----------------------------------------------------------------------------------------
%	PUBLICATIONS
%----------------------------------------------------------------------------------------

\begin{rSection}{Publications}

\begin{enumerate}

    \item \textbf{Hengyang Zhao}, Zhongdong Qi, Shujuan Wang, Kambiz Vafai, Hai
        Wang, Haibao Chen, and Sheldon X.-D. Tan ``Learning-Based Occupancy
        Behavior Detection for Smart Buildings.'' International Symposium on
        Circuits and Systems ('16 Invited, Accepted)

    \item Wandi Liu, Hai Wang, \textbf{Hengyang Zhao}, Shujuan Wang, Haibao
        Chen, Yuzhuo Fu , Jian Ma , Xin Li , Sheldon X.-D. Tan ``Thermal
        Modeling for Energy-Efficient Smart Building With Advanced Overfitting
        Mitigation Technique.'' Asia and South Pacific Design Automation
        Conference, pp.  417-422.

    \item \textbf{Zhao, Hengyang}, Daniel Quach, Shujuan Wang, Hai Wang, Haibao
        Chen, Xin Li, and Sheldon X-D. Tan. ``Learning Based Compact Thermal
        Modeling for Energy-Efficient Smart Building Management.'' In
        Proceedings of the IEEE/ACM International Conference on Computer-Aided
        Design, pp.  450-456. IEEE Press, 2015

    \item He, Kai, Sheldon X.-D. Tan, \textbf{Hengyang Zhao}, Xue-Xin Liu, Hai Wang, and
        Guoyong Shi. ``Parallel GMRES solver for fast analysis of large linear
        dynamic systems on GPU platforms.'' Integration, the VLSI Journal 52
        (2016): 10-22.

\end{enumerate}

\end{rSection}

\end{document}
